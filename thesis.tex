\documentclass[12pt,a4paper,oneside,open=right,bibliography=totoc,BCOR=10mm]{scrreprt} % Schriftgröße, Seitenformat, Zweiseitig für Seitenränder, Bindcorrection

\setlength{\parindent}{1em}
%\setlength{\parskip}{1em}
%\renewcommand{\baselinestretch}{1.5}

\usepackage[utf8]{inputenc} % Richtiges anzeigen von Umlauten und quasi allen anderen Schriftzeichen
\usepackage{cmap}
\usepackage[T1]{fontenc} % Wichtig für alles was mehr als ASCII verwendet
\usepackage{csquotes} % Schöne Anführungsstriche mit \enquote{Text}
\usepackage{amsmath} % Bessere und schönere mathematische Formeln
\usepackage{mathtools} % Noch schönerere mathematische Formeln
\usepackage{amstext} % \text{} Macro in mathematischen Formeln
\usepackage{amsfonts} % Erweiterte Zeichensätze für mathematische Formeln
\usepackage{amssymb} % Spezielle mathematische Symbole.
\usepackage{array} % Matrizen in mathematischen Formeln
\usepackage{textcomp} % Für textmu und textohm etc. um im Fließtext keine Mathematik 
\usepackage{textgreek} % Damit können griechische Zeichen direkt im Text verwendet werden (siehe zeichen.txt)
\usepackage{paralist} % Für compactitem und compactenum
\usepackage{xstring} % Für IF in Titelseite

\usepackage{braket} % Für das quantenmechanische Bra-Ket

\usepackage{float}

\usepackage{geometry} % Seitenränder und Seiteneigenschaften setzen
%\usepackage[showframe]{geometry} % Anzeigen der Seitenränder, nützlich für debugging. http://ctan.org/pkg/geometry

\usepackage[bottom]{footmisc} % Zwingt Fußnoten an das Ende der Seite
\usepackage[pdftex]{hyperref} % Links richtig anzeigen. Sowohl innerhalb des Dokuments (Fußzeilen, Formeln), als auch ins Internet

\hypersetup{
	colorlinks=false,
	pdfborder={0 0 0},
}

\usepackage[ % Biblatex für die Zitate und Referenzen
	backend=biber,
	hyperref=true
		]{biblatex}

\usepackage{xkeyval} % Erlaubt "Variablen" zu definieren, wird für Titelseite gebraucht
\usepackage{graphicx} % Wichtig für das Einbinden von Grafiken
\usepackage{caption}
\usepackage{subcaption} % Einbinden von mehreren Grafiken in einer figure

\usepackage{dirtree} % Erlaubt das erstellen von Dateibäumen
% \dirtreecomment{Text} erstellt einen Kommentar zu dem Verzeichnis bzw. der Datei
\newcommand{\dirtreecomment}[1]{\dotfill{} \begin{minipage}[t]{0.5\textwidth}#1\end{minipage}}

\usepackage{fancyvrb} % Mehr Optionen für Verbatim
\usepackage{listings} % Zur Darstellung von Programmcode
\usepackage{pdflscape} % Querformat Seiten

\newcommand{\writeIn}[1]{\usepackage[#1]{babel}} % Definiert einen neuen Befehl um die Sprache des Dokuments zu setzen

\usepackage[usenames,dvipsnames]{color} % Farben für den todo Befehl
\newcommand{\todo}[1]{{\color{Cerulean}(TODO: #1)}} % Einfach \todo{Text} verwenden!

\newcommand{\blankpage}{ \newpage \thispagestyle{empty} \mbox{} \newpage }

\renewcommand*\descriptionlabel[1]{\hspace\leftmargin$#1$}

\definecolor{codegreen}{rgb}{0,0.6,0}
\definecolor{codegray}{rgb}{0.5,0.5,0.5}
\definecolor{codepurple}{rgb}{0.58,0,0.82}
\definecolor{backcolour}{rgb}{0.95,0.95,0.92}
 
\lstdefinestyle{mystyle}{
    backgroundcolor=\color{backcolour},   
    commentstyle=\color{codegreen},
    keywordstyle=\color{magenta},
    numberstyle=\tiny\color{codegray},
    stringstyle=\color{codepurple},
    basicstyle=\footnotesize,
    breakatwhitespace=false,         
    breaklines=true,                 
    captionpos=b,                    
    keepspaces=true,                 
    numbers=left,                    
    numbersep=5pt,                  
    showspaces=false,                
    showstringspaces=false,
    showtabs=false,                  
    tabsize=2
}
 
\lstset{style=mystyle}

\makeatletter
\define@cmdkey{thesisTitlePage}{author}[Max Mustermann]{}
\define@cmdkey{thesisTitlePage}{title}[Titel]{}
\define@cmdkey{thesisTitlePage}{institute}[Institut]{}
\define@cmdkey{thesisTitlePage}{prof}[Professor]{}
\define@cmdkey{thesisTitlePage}{address}[Adresse]{}
\define@cmdkey{thesisTitlePage}{type}[Typ der Arbeit]{}
\newcommand{\setThesisTitlePage}[1]{\setkeys{thesisTitlePage}{#1}}

% Default Werte für die Variablen
\setkeys{thesisTitlePage}{
	author=Max Mustermann,
	title=Titel,
	institute=Institut,
	prof=Professor,
	address=Adresse des Autors,
	type=bacc
}{}

\newcommand*{\thesisTitlePage}{
	\begingroup % Create the command for including the title page in the document
	\newgeometry{bottom=2cm, top=2cm, left=3cm, right=2cm}
	\begin{titlepage}
	
	
	\begin{tabular}{ >{\centering}p{9cm} >{\centering}p{7cm} }
		\space & {\line(1,0){120}\\Signature of Advisor}
	\end{tabular}
	
	\begin{center}
	
	% Upper part of the page
	\begin{figure}[h]
		\centering
			\includegraphics[width=0.5\textwidth]{figures/tu_wien_logo.pdf}
		 %Logo gracefully taken from http:www.tuwien.ac.at/dle/pr/publishing_web_print/corporate_design/tu_logo/
	\end{figure}
	
	\vspace{\stretch{1}}
	\begin{LARGE}
	
	\par\noindent%
	 \IfStrEqCase{\cmdKV@thesisTitlePage@type}{%
	  {sem}{SEMINARARBEIT}%
	  {bacc}{Bachelor thesis}%
	  {proj}{Project thesis}%
	  {mast}{DIPLOMARBEIT}% Laut Auskunft Dekanat muss auch eine Masterarbeit DIPLOMARBEIT heißen
	  {dipl}{DIPLOMARBEIT}%
	  {diss}{DISSERTATION}%
	  }[\cmdKV@thesisTitlePage@type]
	
	\vspace{\stretch{1.8}}
	
	\textbf{\cmdKV@thesisTitlePage@title} \\
	
	\end{LARGE}
	
	\vspace{\stretch{1.8}}
	\begin{large}
	\cmdKV@thesisTitlePage@institute \\
	TU Wien
	
	\vspace{\stretch{0.5}}
	
	Supervisors: \\
	\textbf{\cmdKV@thesisTitlePage@prof}
	
	\vspace{\stretch{1}}
	
	by \\
	
	\vspace{\stretch{0.3}}
	
	\textbf{\cmdKV@thesisTitlePage@author} \\
	
	\vspace{\stretch{0.3}}
	
	\cmdKV@thesisTitlePage@address \\
	
	\vspace{\stretch{2}}
	
	\begin{tabular}{ >{\centering}p{7cm} >{\centering}p{7cm} }
	\centering
	\today & \line(1,0){120}\\Signature of Author
	\end{tabular}
	\end{large}
	
	\end{center}
	\end{titlepage}
	\restoregeometry
	\endgroup
}

\newcommand*{\thesisTitlePageBlank}{
	\begingroup % Create the command for including the title page in the document
	\newgeometry{bottom=2cm, top=2cm, left=3cm, right=2cm}
	\begin{titlepage}
	\begin{center}
	\vspace{\stretch{1}}
	\begin{LARGE}
	\vspace{\stretch{1.8}}
	\textbf{\cmdKV@thesisTitlePage@title} \\
	\end{LARGE}
	\vspace{\stretch{1.8}}
	\begin{large}
	by \\
	\vspace{\stretch{0.3}}
	\textbf{\cmdKV@thesisTitlePage@author}
	\end{large}
	\end{center}
	\end{titlepage}
	\restoregeometry
	\endgroup
}
\makeatother

\writeIn{english} % Siehe header.tex. Setzt Dokumentsprache und damit Sprache von "Abstract", "Inhaltsverzeichnis", Datumsangaben etc.

\hypersetup{ % Setzt einige Werte die in den Eigenschaften des PDF gespeichert sind.
	pdfauthor={Andreas Stefl},
	pdftitle={DIIS mixing of self energy in DMFT},
	pdfsubject={Project thesis physics},
	pdfkeywords={},
	pdfdisplaydoctitle=true,
	colorlinks=false, % Für Druck auf "false" setzen!
}
\addbibresource{references.bib}
\bibliographystyle{unsrt}
\begin{document}

\setThesisTitlePage{
	title={DIIS mixing of self energy in DMFT},
	institute={Institute of Solid State Physics},
	prof={Dipl.-Ing. Josef Kaufmann\\Univ. Prof. Dr. Karsten Held},
	author={Andreas Stefl},
	address={Wien},
	type={proj},
	% Vordefinierte Typen sind: sem (Seminararbeit), bacc (Bachelorarbeit), proj (Projektarbeit), mast (Diplomarbeit), dipl (Diplomarbeit), diss (Dissertation)
	% Laut Auskunft des Dekants muss auch eine Masterarbeit den Titel "Diplomarbeit" tragen.
	% Bei allen anderen Typen werden die Texte direkt übernommen.
}

\pagenumbering{gobble} % Keine Seitenzahl drucken
\thesisTitlePage

%4 sentences
%state the problem
%say why it's an interesting problem
%say what your solution achieves
%say what follows from your solution





\tableofcontents \newpage
\cleardoublepage % Macht, dass openright funktioniert.
% \chapter macht das automatisch, \tableofcontents und \printbibliography machen das nicht.
% Falls es Probleme gibt hilft auch der Befehl \blankpage (siehe header.tex)
\pagenumbering{arabic} \setcounter{page}{1}

%%problem statement (which problem should be solved?)
%aim of the work
%methodological approach
%structure of the work

\chapter{Motivation}
\label{ch:motivation}
% TODO cluster computing "cpu time = money"
% TODO diis popular in dga?
% TODO convergence: -> possible speed up, -> might enable convergence


\chapter{Theoretical Background}
\label{ch:background}

\section{DMFT}
Dynamical mean-field theory (DMFT)\cite{dmft} is a method to determine the electronic structure of strongly correlated materials. It consists of a self-consistency iteration \(\varphi(x^\ast) = x^\ast\), shown in figure \ref{fig:dmft}, with a very complicated, non-linear function \(\varphi\). $x$ is the self-engery $\Sigma$ and its moments $\Sigma_0$ and $\Sigma_1$.

\begin{figure}[H]
    \centering
    \includegraphics[width=1.0\textwidth]{figures/dmft.png}
    \caption{Self-consistency loop of dynamical mean field theory for a Hubbard-type model. Image taken from Markus Wallerberger's dissertation.\cite{wallerberger2016}}
    \label{fig:dmft}
\end{figure}

\section{Mixing}
Mixing is a technique to improve convergence for self-consistency iteration. The idea is to construct a more effective trial with more information than just the previous result. To be effective such a trial has to be closer to the true value which means we want to extrapolate in terms of iterations.

From now on we will use $x_k$ as the $k$th iteration trial, $\varphi(x_k)$ as the $k$th iteration result and $x^\ast$ as the true value.

The simplest approach is linear mixing
\begin{equation} \label{eq:linmix}
x_{k+1} = (1-\beta) x_k + \beta \varphi(x_k) = x_k + \beta (\varphi(x_k) - x_k)
\end{equation}
with the relaxation parameter \(\beta\).
For \(0 < \beta < 1\) we get an under-relaxation which slows down convergence in some cases, but also dampens oscillations. If the self-consistency iterations diverge, the use of under-relaxation might help.
For \(\beta > 1\) we get over-relaxation which can have predictive behaviour and speed up the convergence but it also amplifies errors and might prevent the convergence of the iteration scheme.

More complex mixing techniques use trials and results of multiple previous iterations and try to combine under-relaxation for robustness and prediction to increase the rate of convergence\cite{anderson_mixing}.

\section{DIIS}
The direct inversion of the iterative subspace (DIIS), also known as Pulay mixing, is an extrapolation technique developed by Peter Pulay\cite{diis_pulay1}\cite{diis_pulay2}. His intention was to accelerate and stabilize the convergence of the Hartree-Fock self-consistent field method. Besides that, DIIS was successfully applied to accelerate the self-consistent field (SCF) approach for density functional theory (DFT) calculations.\cite{diis_restarted}

DIIS uses trials and results of multiple previous iterations, constructs a linear combination of them, and extrapolates a new trial for the next iteration. The coefficients are determined by a least squares optimization.

Pulay's DIIS attempts to construct a better trial by using multiple previous trials and results. The method assumes that a good approximation of the true value \(x^\ast\) can be obtained by a linear combination of the previous trials. That is,
\begin{equation} \label{eq:diis_x}
\overline{x}_{k} = \sum_{j=0}^{m} c_j^k x_{k-j}
\end{equation}
where \(m\) is the number of previous trials to consider. We can also write this sum in terms of the true value \(x^\ast\) and an error vector \(e_{k} = x^\ast - x_{k}\).
\[\overline{x}_{k} = \sum_{j=0}^{m} c_j^k (x^\ast + e_{k-j}) = x^\ast \sum_{j=0}^{m} c_j^k + \sum_{j=0}^{m} c_j^k e_{k-j}\]
To get close to the true value \(x^\ast\) we set \(\sum_{j=0}^{m} c_j^k = 1\) and try to minimize the second term \(\sum_{j=0}^{m} c_j^k e_{k-j}\). But since we do not know the true value \(x^\ast\) we cannot know \(e_{k}\). Therefore, we approximate \(e_{k}\) by the residuals, i.e. \(e_{k} \approx f_{k} = g(x_k) - x_k\):
\begin{equation}
\overline{f}_{k} = \sum_{j=0}^{m} c_j^k f_{k-j}
\end{equation}
In order to minimize \(\overline{f}_{k}\) we use the \(l^2\)-norm.

One solution would be to use the Lagrange multiplier technique to satisfy the constraint \(\sum_{j=0}^{m} c_j^k = 1\) and minimize \(|\overline{f}_{k}|^2\).\cite{diis_math}

However, we can embed the constraint by rewriting \footnotemark
\begin{equation} \label{eq:diis_f}
\overline{f}_{k} = \sum_{j=0}^{m} c_j^k f_{k-j} = f_k - \sum_{j=1}^{m} \gamma_j^k {\Delta f}_{k-m+j} = f_k - F_k \Gamma_k
\end{equation}
with the column vector \({\Delta f}_{k} = f_k - f_{k-1}\), the matrix \(F_k = [{\Delta f}_{k-m+j}]_{j=1..m}\) and the column vector \(\Gamma_k = [\gamma_j^k]_{j=1..m}\).

\footnotetext{For example with \(m=2\) we get \(\overline{f}_{k} = f_k - \gamma_1^k {\Delta f}_{k-1} - \gamma_2^k {\Delta f}_{k}\ = f_k - \gamma_1^k (f_{k-1} - f_{k-2}) - \gamma_2^k (f_k - f_{k-1}) = (1-\gamma_2^k) f_k + (\gamma_2^k - \gamma_1^k) f_{k-1} + \gamma_1^k f_{k-2} = c_0 f_k + c_1 f_{k-1} + c_2 f_{k-2}\). Equating coefficients leads to \(\sum_{j=0}^{m} c_j^k = 1\).}

We minimize \(|\overline{f}_{k}|^2\) by \(\Gamma_k\) and get the solution \footnotemark
\begin{equation} \label{eq:diis_gamma}
\Gamma_k = (F_k^\dagger F_k)^{-1} F_k^\dagger f_k
\end{equation}

\footnotetext{$\Gamma_k$ only exists if $F_k^\dagger F_k$ is invertible. The Moore–Penrose inverse $F_k^+$ exists even if that is not the case. Which would lead to the solution $\Gamma_k = F_k^+ f_k$.}

Similar to \eqref{eq:linmix} we can construct a new trial with \eqref{eq:diis_x}, \eqref{eq:diis_f} and \eqref{eq:diis_gamma}.
\begin{equation} \label{eq:diis_final}
x_{k+1} = \overline{x}_k + \beta \overline{f}_k = x_k + \beta f_k - (X_k + \beta F_k) (F_k^\dagger F_k)^{-1} F_k^\dagger f_k
\end{equation}
with the matrix \(X_k = [{\Delta x}_{k-m+j}]_{j=1..m}\) and the column vector \({\Delta x}_{k} = x_k - x_{k-1}\).

One interpretation of \eqref{eq:diis_final} is that \((F_k^\dagger F_k)^{-1} F_k^\dagger\) is a regression to predict the transformed weights $\Gamma_k$. The regression is trained with \(m\) previous observations and operates on the current observations \(f_k\). $\beta$ decides if we mix trials (\(\beta = 0\)) or results (\(\beta = 1\)).

Variations of the DIIS method exist to further increase the rate of convergence and stability. The restarted Pulay method\cite{diis_restarted} and the periodic Pulay method\cite{diis_periodic} reset the state or use linear mixing in between to reduce side effects of DIIS.


%\chapter{Method}
\label{ch:method}

\section{w2dynamic}
% TODO what is it, who wrote it
% TODO qmc background?

\section{Cluster computing}
% TODO general, hclm, vsc
% TODO why do we need a cluster

\section{Programming language}
% TODO python, numpy, matplotlib

\section{Tools}
% TODO git, hgrep


\chapter{Implementation}
\label{ch:impl}

\section{DIIS}

\begin{lstlisting}[label=lst:diis, language=python, caption=DIIS implementation]
class DiisMixer(object):
    def __init__(self, oldshare, history, period):
        self.alpha = 1 - oldshare
        self.history = history
        self.period = period

        self.i = 0
        self.trials = []
        self.residuals = []

    def __call__(self, new_value):
        if self.i == 0:
            # no history yet
            result = new_value
        else:
            trial = self.trials[-1]
            residual = new_value - trial
            self.residuals.append(residual)

            # trim history
            self.trials = self.trials[:self.history]
            self.residuals = self.residuals[:self.history]

            if self.i <= 2 or (self.i % self.period) != 0:
                # linear mixing
                result = trial + self.alpha * residual
            else:
                # pulary mixing
                R = np.array(self.trials); R = R[1:] - R[:-1]; R = R.T
                F = np.array(self.residuals); F = F[1:] - F[:-1]; F = F.T
                result = trial + self.alpha * residual \
                         - np.linalg.multi_dot([R + self.alpha * F, np.linalg.pinv(F), residual])

        self.i += 1
        self.trials.append(result)
        return result
\end{lstlisting}

% TODO why pinv? -> equality but also exists if not invertible
This DIIS mixing implementation works with real vectors. It is similar to the periodic Pulay method\cite{diis_periodic}. The most significant difference is the use of the pseudo inverse.

\section{Refactor}
% TODO mention or show linear mixing refactor?

% TODO why refactor and what -> mix all quantities in one step -> why
blabla

% TODO explain/cite decorator pattern and why it fits
The quantities we want to mix are not vectors but higher dimensional objects with complex numbers. To use our DIIS implementation in this case, we can naively reshape the data to a vector and seperate the real and imaginary part of the complex numbers. This functionality was implemented with the Decorator pattern.

\begin{lstlisting}[label=lst:flatdec, language=python, caption=Flat mixing decorator]
class FlatMixingDecorator(object):
    def __init__(self, mixer):
        self.mixer = mixer
    def __call__(self, *args):
        if len(args) == 1: args = args[0]
        types = deepflatten.types(args)
        shape = deepflatten.shape(args)
        x = deepflatten.flatten(args)
        x = self.mixer(x)
        x = deepflatten.restore(x, shape, types)
        return x
\end{lstlisting}

\begin{lstlisting}[label=lst:realdec, language=python, caption=Real mixing decorator]
class RealMixingDecorator(object):
    def __init__(self, mixer):
        self.mixer = mixer
    def __call__(self, x):
        n = x.shape[0]
        x = np.concatenate([np.real(x), np.imag(x)])
        x = self.mixer(x)
        x = x[:n] + 1j*x[n:]
        return x
\end{lstlisting}

\section{Deep flatten}
% TODO why deep flatten -> input is a mix of numbers, python lists and numpy arrays with different data types -> generalization desireable to avoid input constraints (wide contract)
% TODO explain functions, mention recursion caused by nested lists

% TODO move to appendix?
\begin{lstlisting}[label=lst:deepflat, language=python, caption=deepflatten.py]
import numbers
import numpy as np

def types(x):
    if type(x) is np.ndarray: return x.dtype
    if isinstance(x, numbers.Number): return type(x)
    result = []
    for i in x: result.append(types(i))
    return result

def shape(x):
    if type(x) is np.ndarray: return x.shape
    if isinstance(x, numbers.Number): return 1
    result = []
    for i in x: result.append(shape(i))
    return result

def flatten(x):
    if type(x) is np.ndarray: return x.flatten()
    if isinstance(x, numbers.Number): return np.array([x])
    result = np.array([])
    for i in x: result = np.append(result, flatten(i))
    return result

def restore(x, shape, types):
    def recursive(x, shape, types):
        if isinstance(shape, tuple):
            size = np.prod(shape)
            return x[:size].reshape(shape).astype(types), size
        if shape == 1:
            return types(x[0]), 1
        result = []
        offset = 0
        for s, t in zip(shape, types):
            tmp = recursive(x[offset:], s, t)
            result.append(tmp[0])
            offset += tmp[1]
        return result, offset
    return recursive(x, shape, types)[0]
\end{lstlisting}


\chapter{Evaluation}
\label{ch:evaluation}
The hclm cluster was used to evaluate the performance of DIIS for mixing the self-energy. We decided to use reasonable small calculations that would finish in a few hours.

The chosen hamilton was given by a hubbard square with $80\times80$ sites, half filling, $U=8$ and $\beta=50$.

\lstdefinelanguage{ini}{
  basicstyle=\ttfamily\small,
  columns=fullflexible,
  morecomment=[s][\color{Orchid}\bfseries]{[}{]},
  morecomment=[s][\color{Orchid}\bfseries]{[[}{]]},
  morecomment=[l]{\#},
  morecomment=[l]{;},
  commentstyle=\color{gray}\ttfamily,
  morekeywords={},
  keywordstyle={\color{green}\bfseries}
}

\begin{lstlisting}[label=lst:w2dyn_config, language=ini, caption=The w2dynmaics configuration for this case]
[General]
DOS=ReadIn
HkFile=hubbard_2d_80_80_1.hk
beta=50.
NAt=1
totdens=1.
EPSN=0
mu=4.0
DMFTsteps=50
StatisticSteps=0
FileNamePrefix=square_u08_b50_diis
magnetism=para
siw_moments=estimate
FTType=none
mixing=0.0
mixing_strategy=diis
mixing_diis_history=5
mixing_diis_period=1

[Atoms]
[[1]]
Hamiltonian=Density
Udd=8.0
Nd=1

[QMC]
Nwarmups=1e6
Nmeas=1e5
NCorr=200
Ntau=1000
Niw=2000
Eigenbasis=1 
MeasGiw=1
\end{lstlisting}

The baseline was given by the same configuration, but without DIIS mixing.

\begin{figure}[H]
    \centering
    \includegraphics[width=1.0\textwidth]{figures/square_u08_b50_siw_imag.png}
    \caption{DMFT iteration comparison for matsubara frequency 0}
    \label{fig:dmft}
\end{figure}


\chapter{Conclusion and Outlook}
\label{ch:conclusion_outlook}
So far, only ''small'' DMFT calculations were used to evaluate the DIIS mixing. For these cases the rate of convergence is already high and the mixing does not further improve it. The more interesting case of ''big'' calculations needs further investigation.

Overall DIIS seems to be applicable for the self-consistency iteration of DMFT.

DIIS is able to handle complex quantities. The current implementation transforms the complex numbers into a larger array of real values. That might bias the estimation because this is only one way to represent complex numbers. Further investigations are necessary.

DMFT with QMC is generating noisy outputs and seems to challenge DIIS' robustness. It might be possible to modify the DIIS algorithm to handle noise in a better way. One way is to replace the least squares optimization by a weighted least squares optimization. The self-energy has an increasing uncertainty with higher frequencies but all frequencies are treated equally right now. A second way would be to look into the singular value decomposition of $F$ and remove singular values which only represent noise.

As mentioned in the DIIS section, the algorithm can be interpreted as a regression. Since there are a lot of different regression techniques, especially in the field of machine learing, it might be interesting to try out alternatives. Besides that, the mixing algorithm could take advantage of incorporating prior knowledge from previous calculations.



\cleardoublepage
\pagenumbering{roman} \setcounter{page}{1}
\printbibliography

\end{document}

